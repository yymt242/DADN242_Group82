\newpage
\section{USE CASE CHI TIẾT VÀ SƠ ĐỒ HOẠT ĐỘNG, SƠ ĐỒ LỚP CỦA HỆ THỐNG}
\subsection{Sơ đồ Use Case của hệ thống}

Manager (admin)

- Điều chỉnh các thiết bị trong phòng học
- User
- Giáo viên
- Học sinhg



\subsection{Đặc tả Use Case của hệ thống}

\begin{longtable}{|p{0.2\textwidth}|p{0.8\textwidth}|}
    \hline
    \textbf{Use Case}         & \textbf{Quản lý hệ thống - Thiết lập ngưỡng cảm biến}                                                       \\
    \hline
    \textbf{Actors}           & Người quản lý hệ thống                                                                                      \\
    \hline
    \textbf{Descriptions}     & Cho phép người quản lý hệ thống tạo ngưỡng giá trị của các cảm biến trong hệ thống báo động nhà thông minh. \\
    \hline
    \textbf{Precondition}     &
    - Người quản lý hệ thống đã đăng nhập vào ứng dụng. \newline
    - Hệ thống cảm biến đang hoạt động và có thể nhận dữ liệu.                                                                              \\
    \hline
    \textbf{Normal flow}      &
    1. Người quản lý hệ thống mở ứng dụng và chọn "Cấu hình cảm biến". \newline
    2. Hệ thống hiển thị danh sách các cảm biến có thể thiết lập ngưỡng. \newline
    3. Người quản lý chọn một cảm biến cụ thể (nhiệt độ, độ ẩm, ánh sáng, chuyển động). \newline
    4. Người quản lý nhập giá trị ngưỡng tối đa và tối thiểu cho cảm biến. \newline
    5. Hệ thống lưu giá trị ngưỡng vào cơ sở dữ liệu. \newline
    6. Hệ thống xác nhận rằng thiết lập đã được cập nhật thành công.                                                                        \\
    \hline
    \textbf{Alternative flow} &
    \textbf{Alternative 1 ở Bước 4:} \newline
    - Nếu giá trị nhập vào vượt quá giới hạn hợp lệ (ví dụ: nhiệt độ ngưỡng vượt quá 100°C), hệ thống sẽ báo lỗi và yêu cầu nhập lại.       \\
    \hline
    \textbf{Exceptions}       &
    \textbf{Exception ở Bước 2:} \newline
    - Nếu hệ thống không thể kết nối với cơ sở dữ liệu, hệ thống sẽ hiển thị thông báo lỗi và không thể thực hiện thao tác. \newline
    \textbf{Exception ở Bước 6:} \newline
    - Nếu xảy ra lỗi khi lưu giá trị ngưỡng, hệ thống sẽ yêu cầu thử lại hoặc kiểm tra kết nối.                                             \\
    \hline
    \caption{Use Case 1: Thiết lập ngưỡng cảm biến}
    \label{tab:usecase1}
\end{longtable}

\begin{longtable}{|p{0.2\textwidth}|p{0.8\textwidth}|}
    \hline
    \textbf{Use Case}         & \textbf{Quản lý hệ thống - Xem ngưỡng cảm biến}                                                       \\
    \hline
    \textbf{Actors}           & Người quản lý hệ thống                                                                                \\
    \hline
    \textbf{Descriptions}     & Cho phép quản lý hệ thống xem ngưỡng giá trị của các cảm biến trong hệ thống báo động nhà thông minh. \\
    \hline
    \textbf{Precondition}     &
    - Người quản lý hệ thống đã đăng nhập vào ứng dụng. \newline
    - Hệ thống cảm biến đang hoạt động và có thể xuất dữ liệu.                                                                        \\
    \hline
    \textbf{Normal flow}      &
    1. Người quản lý hệ thống mở ứng dụng và chọn "Cấu hình cảm biến". \newline
    2. Hệ thống hiển thị danh sách các cảm biến và giá trị ngưỡng hiện tại. \newline
    3. Người quản lý kiểm tra thông tin và có thể lựa chọn chỉnh sửa hoặc xóa. \newline
    4. Người quản lý thoát màn hình xem ngưỡng hoặc quay lại màn hình chính.                                                          \\
    \hline
    \textbf{Alternative flow} &
    \textbf{Alternative 1 ở Bước 3:} \newline
    - Nếu không có dữ liệu ngưỡng được thiết lập trước đó, hệ thống sẽ hiển thị thông báo "Chưa có ngưỡng nào được thiết lập".        \\
    \hline
    \textbf{Exceptions}       &
    \textbf{Exception ở Bước 2:} \newline
    - Nếu hệ thống không thể kết nối với cơ sở dữ liệu, hệ thống sẽ hiển thị thông báo lỗi và không thể thực hiện thao tác.           \\
    \hline
    \caption{Use Case 2: Xem ngưỡng cảm biến}
    \label{tab:usecase2}
\end{longtable}

% Use Case 3
\begin{longtable}{|p{0.2\textwidth}|p{0.8\textwidth}|}
    \hline
    \textbf{Use Case}         & \textbf{Quản lý hệ thống - Xóa ngưỡng cảm biến}                                                                   \\
    \hline
    \textbf{Actors}           & Người quản lý hệ thống                                                                                            \\
    \hline
    \textbf{Descriptions}     & Cho phép quản lý hệ thống tạo, xem và xóa ngưỡng giá trị của các cảm biến trong hệ thống báo động nhà thông minh. \\
    \hline
    \textbf{Precondition}     &
    - Người quản lý hệ thống đã đăng nhập vào ứng dụng. \newline
    - Hệ thống cảm biến đang hoạt động và có thể nhận dữ liệu.                                                                                    \\
    \hline
    \textbf{Normal flow}      &
    1. Người quản lý hệ thống mở ứng dụng và chọn "Cấu hình cảm biến". \newline
    2. Hệ thống hiển thị danh sách các cảm biến có thể thiết lập. \newline
    3. Người quản lý chọn một cảm biến cụ thể để xóa ngưỡng. \newline
    4. Hệ thống yêu cầu xác nhận trước khi xóa. \newline
    5. Người quản lý xác nhận xóa ngưỡng. \newline
    6. Hệ thống xóa dữ liệu ngưỡng khỏi cơ sở dữ liệu. \newline
    7. Hệ thống hiển thị thông báo xác nhận xóa thành công.                                                                                       \\
    \hline
    \textbf{Alternative flow} &
    \textbf{Alternative 1 ở Bước 5:} \newline
    - Nếu quản lý không xác nhận xóa, hệ thống sẽ hủy thao tác và quay lại màn hình trước đó.                                                     \\
    \hline
    \textbf{Exceptions}       &
    \textbf{Exception ở Bước 2:} \newline
    - Nếu hệ thống không thể kết nối với cơ sở dữ liệu, hệ thống sẽ hiển thị thông báo lỗi và không thể thực hiện thao tác.                       \\
    \hline
    \caption{Use Case 3: Xóa ngưỡng cảm biến}
    \label{tab:usecase3}
\end{longtable}


% Use Case 4
\begin{longtable}{|p{0.2\textwidth}|p{0.8\textwidth}|}
    \hline
    \textbf{Use Case}         & \textbf{Theo dõi thông tin cảm biến}                                                                             \\
    \hline
    \textbf{Actors}           & Người quản lý hệ thống                                                                                           \\
    \hline
    \textbf{Descriptions}     & Hệ thống cho phép quản lý hệ thống theo dõi, thống kê và xuất báo cáo dữ liệu cảm biến theo từng tháng hoặc quý. \\
    \hline
    \textbf{Precondition}     &
    - Quản lý hệ thống đã đăng nhập vào ứng dụng. \newline
    - Dữ liệu cảm biến đã được ghi nhận trong hệ thống.                                                                                          \\
    \hline
    \textbf{Normal flow}      &
    1. Quản lý hệ thống mở ứng dụng và chọn "Thống kê cảm biến". \newline
    2. Hệ thống hiển thị các tùy chọn thời gian: tháng/quý/năm. \newline
    3. Quản lý chọn khoảng thời gian cần xem (ví dụ: tháng 1 năm 2025, quý 1 năm 2025). \newline
    4. Hệ thống truy xuất dữ liệu cảm biến tương ứng trong khoảng thời gian đó. \newline
    5. Hệ thống hiển thị biểu đồ, số liệu thống kê về mức độ thay đổi của các cảm biến (nhiệt độ, độ ẩm, ánh sáng, chuyển động). \newline
    6. Quản lý có thể chọn xem chi tiết từng chỉ số cảm biến. \newline
    7. Quản lý thoát màn hình thống kê hoặc tiếp tục thao tác khác.                                                                              \\
    \hline
    \textbf{Alternative flow} &
    \textbf{Alternative 1 ở Bước 3:} \newline
    - Nếu không có dữ liệu trong khoảng thời gian đã chọn, hệ thống hiển thị thông báo "Không có dữ liệu cảm biến nào được ghi nhận trong khoảng thời gian này". \newline
    \textbf{Alternative 2 ở Bước 5:} \newline
    - Quản lý có thể lọc dữ liệu theo loại cảm biến cụ thể (chỉ hiển thị nhiệt độ hoặc chỉ hiển thị độ ẩm).                                      \\
    \hline
    \textbf{Exceptions}       &
    \textbf{Exception ở Bước 4:} \newline
    - Nếu hệ thống không thể truy xuất dữ liệu từ cơ sở dữ liệu, hệ thống hiển thị thông báo lỗi và yêu cầu thử lại.                             \\
    \hline
    \caption{Use Case 4: Theo dõi thông tin cảm biến}
    \label{tab:usecase4}
\end{longtable}

% Use Case 5
\begin{longtable}{|p{0.2\textwidth}|p{0.8\textwidth}|}
    \hline
    \textbf{Use Case}         & \textbf{Thống kê thông tin cảm biến}                                                                             \\
    \hline
    \textbf{Actors}           & Người quản lý hệ thống                                                                                           \\
    \hline
    \textbf{Descriptions}     & Hệ thống cho phép quản lý hệ thống theo dõi, thống kê và xuất báo cáo dữ liệu cảm biến theo từng tháng hoặc quý. \\
    \hline
    \textbf{Precondition}     &
    - Quản lý hệ thống đã đăng nhập vào ứng dụng. \newline
    - Dữ liệu cảm biến đã được ghi nhận trong hệ thống.                                                                                          \\
    \hline
    \textbf{Normal flow}      &
    1. Quản lý hệ thống chọn "Xuất báo cáo" trong màn hình thống kê. \newline
    2. Hệ thống cho phép chọn định dạng báo cáo (PDF, CSV, Excel). \newline
    3. Hệ thống tổng hợp dữ liệu và tạo file báo cáo. \newline
    4. Quản lý tải xuống báo cáo hoặc gửi báo cáo qua email. \newline
    5. Hệ thống hiển thị thông báo hoàn tất quá trình xuất báo cáo. \newline
    6. Quản lý thoát màn hình hoặc tiếp tục thao tác khác.                                                                                       \\
    \hline
    \textbf{Alternative flow} &
    \textbf{Alternative 1 ở Bước 3:} \newline
    - Quản lý có thể chọn khoảng thời gian tùy chỉnh (ví dụ: từ ngày 10/01/2025 đến 15/03/2025).                                                 \\
    \hline
    \textbf{Exceptions}       &
    \textbf{Exception ở Bước 3:} \newline
    - Nếu quá trình tạo báo cáo thất bại, hệ thống sẽ hiển thị thông báo lỗi và cho phép thử lại sau.                                            \\
    \hline
    \caption{Use Case 5: Thống kê thông tin cảm biến}
    \label{tab:usecase5}
\end{longtable}

\subsection{Sơ đồ hoạt động của hệ thống}
User | System | Other server

Đăng ký/Xóa lịch sd phòng

Chính tay thiết bị

Thiết lập thời gian tắt cho thiết bị

Tự dộng bật/tắt thiết bị

Thêm xóa sửa thie1t bị

Sửa hồ sơ người dùng

Điều khiển thiết bị bằng giọng nói (AI)

Kiểm tra số lượng người trong phòng (AI)

Validate input, notify user

Xem thông tin phòng

Xem history dùng phòng

Gửi phản hồi

Thêm sửa xóa account
\subsection{Sơ đồ lớp của hệ thống}
- View
- COntroller
- External API
- Model
- Database