\newpage
\section{KẾT LUẬN}

Trong đồ án này, nhóm đã xây dựng thành công một hệ thống cảnh báo an toàn cho gia đình với đầy đủ các tính năng quan trọng:

\begin{itemize}
    \item Theo dõi các chỉ số môi trường như nhiệt độ, độ ẩm, ánh sáng và khoảng cách; đồng thời tự động đưa ra các phản hồi phù hợp như bật quạt, bật đèn LED RGB, đóng cửa.
    \item Điều khiển thiết bị tự động theo ngưỡng giá trị cảm biến với giao diện tuỳ chỉnh mức hoạt động thông qua thanh trượt.
    \item Cảnh báo nguy hiểm: chớp đèn, âm thanh cảnh báo trên web, khoá cửa phần cứng trong trường hợp có đột nhập.
    \item Giao diện dashboard hiển thị dữ liệu thời gian thực với biểu đồ trực quan, cập nhật nhanh, dễ theo dõi.
    \item Hệ thống tài khoản có phân quyền: chỉ quản trị viên mới được tạo tài khoản mới; người dùng có thể đăng nhập, chỉnh sửa và xóa tài khoản.
    \item Đăng nhập/đăng ký mượt mà với xử lý lỗi đầy đủ và tương tác người dùng tối ưu (nhấn Enter sau khi nhập thông tin để đăng nhập, hiển thị hộp thoại thay vì alert thô).
\end{itemize}

Trong tương lai, hệ thống của nhóm có thể được mở rộng và cải tiến theo các hướng sau:

\begin{itemize}
    \item \textbf{Áp dụng trí tuệ nhân tạo (AI): } để học ngưỡng hoạt động tối ưu từ dữ liệu lịch sử, giảm cảnh báo giả.
    \item \textbf{Cải tiến giao diện và trải nghiệm người dùng: }bổ sung chế độ tối/sáng, hỗ trợ đa ngôn ngữ.
    \item \textbf{Tích hợp nhiều kênh cảnh báo khác: } ngoài cảnh báo qua web, thì nên tích hợp thêm các kênh cảnh báo khác như email, Zalo, Telegram hoặc ứng dụng di động để tăng khả năng phản hồi trong thực tế.
    \item \textbf{Mở rộng kết nối thiết bị: } để hỗ trợ thêm các phần cứng mới như camera AI, robot tuần tra hoặc thiết bị khóa thông minh.
\end{itemize}
